\documentclass{beamer}
\usepackage{blindtext} % to fill up slides
\usepackage{verbatim}

% Don't delete:
\newif\ifsplit
% It is necessary for switching from one outer theme to the other. By default, a miniframes outer theme is used for the header. Uncomment the following line to switch to a split outer theme:
% \splittrue
% I suggest using the split outer theme if subsections are meaningful, miniframes otherwise.
\usetheme{KI}

\title{A very nice title}
\author{Author}
\date{\today}

\begin{document}

\begin{frame}[noframenumbering, plain]
\titlepage
\end{frame}

\section{Section 1}

\begin{frame}
\frametitle{Slide 1.1}
\blindtext
\end{frame}

\begin{frame}
\frametitle{Slide 1.2}
\blindtext
\end{frame}

\section{Section 2}

\begin{frame}
\frametitle{Slide 2.1}
\blindtext
\end{frame}

\begin{frame}
\frametitle{Slide 2.2}
\blindtext
\end{frame}

\section{Section 3}

\begin{frame}[fragile] % slides with a verbatim environment need to be 'fragile'
\frametitle{Slide 3.1}
A slide with code:
\begin{verbatim}
x <- rnorm(10)
y <- rnorm(10)
plot(x, y)
\end{verbatim}
\end{frame}
% this needs to be on a line on its own, otherwise will throw an error

\begin{frame}[fragile]
    \frametitle{Slide 3.2}

    \alert{Some \texttt{alert} text.}

    \begin{itemize}
        \item One,
        \item Two,
        \item Three:
        \begin{enumerate}
            \item A,
            \item B,
            \item C.
        \end{enumerate}
    \end{itemize}
\end{frame}

\end{document}
